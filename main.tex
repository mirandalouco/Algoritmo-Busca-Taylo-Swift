\documentclass[a4paper,12pt]{article}
\usepackage[utf8]{inputenc}

\usepackage[brazil]{babel}
\usepackage[lmargin=3cm,tmargin=3cm,rmargin=2cm,bmargin=2cm]{geometry}
\usepackage[T1]{fontenc}
\usepackage{amsmath,amsthm,amsfonts,amssymb,dsfont,mathtools}
\usepackage{blindtext}
\usepackage{graphicx} % Required for inserting images
\usepackage{listings}
\usepackage{xcolor}

\definecolor{codegreen}{rgb}{0,0.6,0}
\definecolor{codegray}{rgb}{0.5,0.5,0.5}
\definecolor{codepurple}{rgb}{0.8,0,0.2}
\definecolor{backcolour}{rgb}{.95,.95,1}
\definecolor{backcolour2}{rgb}{255,255,255}

\lstdefinestyle{mystyle}{
    backgroundcolor=\color{backcolour},   
    commentstyle=\color{codegreen},
    keywordstyle=\color{blue},
    numberstyle=\tiny\color{codegray},
    stringstyle=\color{codepurple},
    basicstyle=\ttfamily\footnotesize,
    breakatwhitespace=false,         
    breaklines=true,                 
    captionpos=b,                    
    keepspaces=true,                 
    numbers=left,                    
    numbersep=5pt,                  
    showspaces=false,                
    showstringspaces=false,
    showtabs=false,                  
}
\lstdefinestyle{mystyle2}{
    backgroundcolor=\color{backcolour2},   
    commentstyle=\color{red},
    keywordstyle=\color{blue},
    numberstyle=\tiny\color{black},
    stringstyle=\color{purple},
    basicstyle=\ttfamily\small,
    breakatwhitespace=false,         
    breaklines=true,                 
    captionpos=b,                    
    keepspaces=true,                 
    numbers=none,                    
    numbersep=5pt,                  
    showspaces=false,                
    showstringspaces=false,
    showtabs=false,                  
}
\lstset{style=mystyle,mystyle2}
\begin{document}


\begin{center}
\textbf{FATEC RUBENS LARA}

\textbf{CURSO DE CIÊNCIA DE DADOS}

\vspace{3cm}

\textbf{ALGORITMO DE BUSCA DE MÚSICAS POR SIMILARIDADE}

\vspace{3cm}

\textbf{GUSTAVO MIRANDA SILVA}

\textbf{ISABELA VIEIRA DA SILVA CRUZ MARTINS}

\textbf{LUIGI LEONE DUARTE YAMAMOTO LEITE}

\textbf{MANUELA SILVA DE ANDRADE}

\vfill

\begin{flushright}
Santos - São Paulo\\
23/10/2023
\end{flushright}
\end{center}

\begin{figure}{}
\centering
\label{}
\includegraphics[width=14cm,height=2cm]{rodap-4.png}
\end{figure}

\clearpage

Neste presente relatório queremos demonstrar um algoritmo de busca que se baseia no ângulo do cosseno e em como ele pode ser aplicado no âmbito musical, de forma que o usuário possa encontrar uma música de acordo com palavras chaves que o mesmo entregar ao algoritmo, essas palavras chaves podem ser utilizadas de maneira que o usuário expresse seus sentimentos para se confortar em alguma música, em ambientes sociais onde é desejável musicas animadas, em estudos onde se utiliza músicas mais calmas para otimização da concentração etc.\\

Variáveis como essas mudam de pessoa a pessoa, se tornam dependente ao usuário e como ele vai se aproveitar do algoritmo, para realização deste relatório, usaremos um dataset de um artista específico com grande quantidade de músicas em sua discografia, que aborde temas emocionais e pessoais, que seu público tenha alta identificação em suas músicas e tenha trabalhado em diferentes gêneros musicais que resulta em uma variadade de tipos músicas para estudo.\\

Para isso foi selecionada a artista norte-americana Taylor Swift que por si acaba satisfazendo os critérios, explorou vários gêneros musicais ao longo de sua carreira, e o estilo de cada álbum reflete sua evolução como artista.\\ 

Aqui está uma breve descrição dos gêneros associados a cada um de seus álbuns:\\\\
"Taylor Swift" (2006) - 11 faixas, principalmente country e country pop.\\
"Fearless" (2008) - 13 faixas, uma mistura de country, country pop e elementos de música pop \\
"Speak Now" (2010) - 14 faixas, explorando o country, o country pop, com elementos de música pop e rock.\\
"Red" (2012) - 16 faixas, incluindo uma variedade de gêneros, como pop, country, rock e indie.\\
"1989" (2014) - 13 faixas, com um som pop mais eletrônico, influências do synth-pop e dance-pop.\\
"Reputation" (2017) - 15 faixas, principalmente música pop, com elementos de hip-hop e eletrônico.\\
"Lover" (2019) - 18 faixas, um retorno a um som pop suave, com algumas faixas mais experimentais.\\
"Folklore" (2020) - 16 faixas, um estilo indie folk e alternativo com um som mais acústico e reflexivo.\\
"Evermore" (2020) - 15 faixas, continua a explorar o estilo indie folk e alternativo estabelecido em "Folklore".\\\\
Total de 131 músicas e 14 gêneros musicais diferentes\\
\pagebreak


No contexto de busca de músicas em um dataset da Taylor Swift, o algoritmo para encontrar músicas que são mais semelhantes em termos de conteúdo musical em relação a uma frase de consulta. Aqui está como é aplicado este algoritmo a um conjunto de dados de músicas da Taylor Swift:\\

Primeiro, cada música é representada como um vetor em um espaço vetorial, uma maneira comum de fazer isso é usar recursos musicais relevantes, como acordes, letras, tempo, tom e outros, mas como dito anteriormente apenas a letra servirá como critério. Cada música será um vetor nesse espaço. O usuário fornece uma frase de consulta, como forma de se expressar, o que também é representada como um vetor no mesmo espaço vetorial.\\

Cálculo do Ângulo do Cosseno:\\
Para encontrar músicas semelhantes à frase de consulta, você calcula o cosseno do ângulo entre o vetor da música de consulta e os vetores de todas as músicas no dataset. Isso é feito usando a fórmula de similaridade do cosseno:\\

Similaridade (cosine similarity) = (Q . M) / (||Q|| * ||M||)\\

Onde:\\
"Q . M" é o produto escalar entre o vetor da música de consulta Q e o vetor da música M no dataset.\\
"||Q||" é a norma (magnitude) do vetor da música de consulta Q.\\
"||M||" é a norma (magnitude) do vetor da música M no dataset.\\

Classificação das Músicas com Base na Similaridade Mapeada:\\
Após calcular a similaridade do cosseno para todas as músicas em relação à música de consulta e ajustar os valores para estarem mais próximos de 0 ou 1, você pode classificar as músicas com base na "Similaridade Mapeada". Para fazer isso, você pode seguir as seguintes diretrizes:\\

Classificar as músicas em ordem decrescente de "Similaridade Mapeada":\\
Músicas com valores mapeados mais próximos de 1 são consideradas as mais semelhantes à música de consulta e, portanto, devem aparecer no topo da lista classificada.\\
Músicas com valores mapeados mais próximos de 0 são consideradas menos semelhantes à música de consulta e, portanto, devem aparecer mais abaixo na lista classificada.\\\\
Desta forma, as músicas que são mais semelhantes à música de consulta, de acordo com a "Similaridade Mapeada," estarão classificadas mais próximas do topo da lista, enquanto aquelas menos semelhantes estarão classificadas mais próximas do final da lista. Isso permite que o usuário encontre mais facilmente as músicas mais relevantes com base na similaridade ajustada de 0 a 1.\\

Recuperação de Músicas:\\
As músicas mais similares são recuperadas e retornadas como resultados da pesquisa. Isso permite que o usuário encontre músicas da Taylor Swift que são musicalmente semelhantes à frase de consulta.\\


A aplicação da fórmula para calcular o Ângulo entre vetores, foi realizada no python\\

\begin{lstlisting}[language=python, caption=Código referente ao cálculo do Angulo entre Vetores]
from sklearn.feature_extraction.text import TfidfVectorizer
from sklearn.metrics.pairwise import cosine_similarity
import pandas as pd

# Carregando o arquivo CSV em um DataFrame
df = pd.read_csv('caminho do dataset', delimiter=';')

# Coluna que contem as letras das musicas
letras = df['lyric']  

frase_referencia = input("Digite a frase de referencia: ")

# Inicialize o vetorizador TF-IDF
vectorizer = TfidfVectorizer()

# Aplique o vetorizador nas letras das musicas
tfidf_matrix = vectorizer.fit_transform(letras)

# Calcule a similaridade de cosseno entre a frase de referencia e as letras das musicas
similaridades = cosine_similarity(vectorizer.transform([frase_referencia]), tfidf_matrix)

# Obtenha os indices das musicas mais similares em ordem decrescente
indices_mais_similares = similaridades.argsort()[0][::-1]

# Crie um conjunto para rastrear as musicas ja incluidas
musicas_incluidas = set()

i = 0
# Imprima as 10 musicas mais similares e seus valores de similaridade 
while len(musicas_incluidas) < 10 and i < len(indices_mais_similares):
    indice = indices_mais_similares[i]
    musica = df.iloc[indice]['track_title']
    valor = similaridades[0][indice]
    
    # Verifique se a musica ja foi incluida e, se nao, imprima-a
    if musica not in musicas_incluidas:
        musicas_incluidas.add(musica)
        print(f"{len(musicas_incluidas)}. Musica: {musica}, Similaridade (Cosseno): {valor}")
    
    i += 1

\end{lstlisting}


\pagebreak


\begin{center}
\textbf{CONCLUSÃO}    
\end{center}

Podemos analisar com testes como se comporta o algoritmo e o quão efetivo se torna sua busca, testamos o primeiro caso onde queriamos que o algoritmo nos devolvessemos a música nomeada como \textit{"Dress".}, ela possui um contexto onde o eu lírico da música se apaixona por seu melhor amigo, dentro da letra da música, o termo \textit{"Best Friend"} se repete muitas vezes e com isso foram feitas duas consultas como mostrado na figura 1 e 2:\\
Q1 = \textit{"best friend"}\\
Q2 = \textit{"im falling in love for my best friend"} \\

Temos que quanto mais direto sua consulta, maior o valor do ângulo do cosseno terá, pois a consulta Q1 possui termos mais similares em comparação a Q2.\\

Segundo caso, se constitui de consultas formadas por frases que capturam o tema das músicas e acertam em encontra-las de forma que o tema seja correspondente, para isso buscamos duas músicas com suas frases que capturem seus temas como observado na figura 3 e 4.\\

Tema da música "Love Story": Q1 = "A timeless love that defied all odds."

Tema da música "Blank Space": Q2 = "Love, chaos, and a hint of madness."\\

Por fim, temos que como ponto fraco a falta de precisão em certos aspectos, pois este algoritmo usa como comparações apenas a letra das músicas, não reconhecendo acordes, tonalidades, tempos entre outros critérios que otimizariam a precisão da busca. Além disso, o uso de metáforas e analogias em músicas, torna a busca da mesma no algoritmo menos precisa, pois em geral o usuário em sua consulta busca por coisas literais que possa se encaixar em seu humor.




\begin{figure}
Print do console testando consultas:\\\\
    \centering
    \includegraphics{dress.png}
    \caption{Console python}
    \label{fig:enter-label}
    \includegraphics{im falling.png}
    \caption{Console python}
    \label{fig:enter-label}
    \centering
    \includegraphics{lov story.png}
    \caption{Console python}
    \label{fig:enter-label}
        \includegraphics{blank.png}
    \caption{Console python}
    \label{fig:enter-label}
\end{figure}    
 







\end{document}


